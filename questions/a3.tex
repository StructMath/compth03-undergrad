کافی است نشان دهیم $B_1 \times B_2 \leq_m B_2$. تابع زیر را در نظر بگیرید:

$$
f(x) = \begin{cases}
    \text{right}(x) & \text{left}(x) \in B_1 \\
    b & \text{otw.}
\end{cases}
$$
این تابع محاسبه‌پذیر تام است زیرا $B_1$ متناهی و در نتیجه تصمیم‌پذیر است.
و در آن $b \notin B_2$ زیرا می‌دانیم که $B_2$ برابر تمام اعداد طبیعی نیست. اکنون داریم:
$$
x\in B_1 \times B_2 \implies f(x) = \text{right}(x) \in B_2
$$

همچنین اگر $x\notin B_1 \times B_2$ به این خاطر که $\text{left}(x) \notin B_1$
در این صورت داریم
$f(x)=b\notin B_2$.
ولی اگر به این خاطر باشد که $\text{right}(x) \notin B_2$ در این صورت چه $f(x)=b$ و چه $f(x) = \text{right}(x)$ در هر دو صورت $f(x)\notin B_2$، پس در نهایت داریم:
$$
x \in B_1 \times B_2 \iff f(x) \in B_2
$$