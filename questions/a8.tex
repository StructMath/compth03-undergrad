فرض کنید $f$ شمارشی بدون تکرار از $K$ باشد. تعریف کنید
$$ S = \{ n \in \mathbb{N} \mid \exists m > n (f(m) < f(n)) \} $$
بنا به قضیهٔ فرم نرمال $S$ \lr{r.e.} است. ‌‌‌می‌توان دید که $\bar{S}$ نامتناهی است، چرا که در غیر این صورت عددی مانند $N$ وجود خواهد داشت که برای هر $N < n$ خواهیم داشت $f(n) < f(n + 1)$، و در نتیجه، $K$ دارای یک شمارش محاسبه‌پذیر صعودی خواهد بود، که با تصمیم‌ناپذیر بودن آن در تناقض است.

همچنین می‌توان نشان داد که $K \equiv_T S$. برای تعیین عضویت یک $x$ در $K$، ابتدا کوچکترین $\bar{S} \ni N$ را می‌یابیم که $x < f(N)$. بنا به تعریف $S$ خواهیم داشت $x \in K$ اگر و تنها اگر $x \in \{ f(n) \mid n < N \}$.

همچنین $\bar{S}$ نمی‌تواند شامل هیچ مجموعهٔ \lr{r.e.} نامتناهی باشد. با برهان خلف، فرض کنید چنین باشد. می‌دانیم هر مجموعهٔ \lr{r.e.} نامتناهی شامل یک مجموعهٔ تصمیم‌پذیر نامتناهی نیز هست (کافی است بخش‌های صعودی تابع شمارندهٔ آن را در نظر بگیریم.) فرض کنید $R$ چنین مجموعه‌ای باشد. در این صورت با استدلالی مشابه بالا خواهیم داشت $K \le_T R$ که با تصمیم‌ناپذیر بودن $K$ در تناقض است.