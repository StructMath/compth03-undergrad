طبق ترکیب توابع و تابع عام، وجود دارد یک $z_0$
به طوری که:
$$
\varphi_{f( s(x,s'(x,y) )}(z) = \varphi_{z_0}(z,y,x)
$$

در اینجا از $s$ و $s'$ استفاده کردیم ولی در ادامه می‌گوییم که آنها چه توابعی هستند، هرچند از تعریف دوری پرهیز کرده‌ایم. طبق قضیه‌ی \lr{s-m-n} وجود دارد یک تابع $s'$ به صورتی که:

$$
\varphi(z,y,x,z_0) = \varphi_{s'(z_0, x)}(z,y)
$$

که در اصل همان تابع \lr{s-m-n} است ولی با یک جایگشت نیز ترکیب شده است. اکنون دوباره طبق  قضیه‌ی \lr{s-m-n}، یک $s$ داریم که:

$$
\varphi(z,y,x,z_0) = \varphi_{s'(z_0, x)}(z,y) = \varphi_{s(y,s'(z_0, x))}(z)
$$

به طور مشابه یک $z_1$ داریم که:
$$
\varphi_{g( s(x,s'(x,y) )}(z) = \varphi_{z_1}(z,y,x)
$$

و همین طور به طور مشابه داریم:

$$
\varphi(z,y,x,z_1) = \varphi_{s'(z_0, x)}(z,y) = \varphi_{s(y,s'(z_1, x))}(z)
$$


اکنون برای معادله اول قرار می‌دهیم $y=z_0$ و $x=z_1$ و در معادله‌ی دوم، برعکس آن را قرار می‌دهیم. در نتیجه داریم:

$$
\varphi_{f( s(z_1,s'(z_1,z_0) )}(z) = \varphi_{s(z_0,s'(z_0, z_1))}(z)
$$

$$
\varphi_{g( s(z_0,s'(z_0,z_1) )}(z) = \varphi_{s(z_1,s'(z_1, z_0))}(z)
$$

در آخر داریم:

$$
e_1 = s(z_1,s'(z_1,z_0) 
$$

$$
e_2 = s(z_0,s'(z_0,z_1)
$$