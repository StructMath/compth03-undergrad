قضیه \lr{(MRDP)}:
$A\subseteq \mathbb{N}$ \lr{r.e.} است، اگر و تنها اگر مجموعه مقادیر نامنفی یک چندجمله‌ای با ضرایب صحیح باشد. یعنی:
$$
x\in A \iff \exists \bar{x}, p(\bar{x}) = x
$$


اثبات قضیه: اگر $A$ مقادیر نامنفی چندجمله‌ای $p(\bar{x})$
باشد
پس:

$$
x\in A \iff \exists \bar{x}, p(\bar{x}) = x \implies A \text{\lr{ is r.e.}}
$$

برعکس اگر $A$ \lr{r.e.} باشد، پس طبق قضیه‌ی ماتیاسویچ 
$q(x, \bar{y})$
چنان موجود است که:
$$
x \in A \iff \exists \bar{y}, q(x, \bar{y}) = 0
$$

حال چندجمله‌ای زیر را در نظر بگیرید:
$$
p(x,\bar{y}) = (x+1)(1-q(x, \bar{y})^2)-1
$$

می‌توان دید که $p(x,\bar{y})$
نامنفی‌ است اگر و تنها اگر $q(x,\bar{y}) = 0$.
حال طبق تعریف $q$
برای $x\in A$ یک $\bar{x}$ هست که 
$q(x,\bar{x}) = 0$
پس
$p(x,\bar{x}) = x$.